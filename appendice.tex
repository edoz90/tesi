\appendix
\begin{appendices}
    \chapter{Satoshi Nakamoto}\label{app:satoshi}
    Satoshi Nakamoto è lo pseudonimo utilizzato dall'utente o dal gruppo di utenti che è stato utilizzato come autore per il white paper \textit{Bitcoin: A Peer-to-Peer Electronic Cash System} e primo sviluppatore della blockchain.\newline
    Il paper fu pubblicato tramite la mailing list di cypherpunk \textit{metzdowd.com} nell'Ottobre del 2008 e la prima versione del software \texttt{bitcoind} fu rilasciata dallo stesso Nakamoto nel Gennaio del 2009.\newline
    L'identità dell'autore è tutt'ora rimasta nascosta ma Satoshi si è presentato come un uomo nato il 5 Aprile 1975 che vive in Giappone. Molte speculazioni sono state fatte sulla possibile identità proponendo personaggi illustri della crittografia moderna o esperti di criptomonete tra cui \texit{Hal Finney}, \texit{Nick Szabo}, \textit{David Chaum}, \textit{Stuart Haber} e \texit{W. Scott Stornetta} in base alle conoscenze espresse da Nakamoto e allo stile di scrittura (analisi stilometrica).\newline
    Satoshi è rimasto attivo nei primi due anni di vita del progetto Bitcoin ma a partire dal 2010, con l'ingresso di nuovi sviluppatore e manutentori le comunicazioni con la community si sono interrotte.\newline
    In quanto all'identià di Nakamoto non è raro che attivisti del movimento cypherpunk vogliano rimanere anonimi ma in aggiunta potrebbe essere una scelta dettata da alcune preoccupazioni come la possibilità di ripercussioni legali per violazione di alcuni brevetti o associazione ad altri episodi di riciclaggio di denario emersi proprio poco prima della pubblicazione del papaer.\newline
    In aggiunta Satoshi ha preferito l'anonimità anche per sicurezza personale e del progetto: è noto che il wallet associato a Nakamoto abbia a disposizione un notevole ammontare di valuta (\texttt{~67} BTC) ed un eventuale cambio in moneta corrente potrebbe far collassare la rete o rendere pubblica la sua identità.
\end{appendices}

% NeoTex: mainfile=main.tex:
