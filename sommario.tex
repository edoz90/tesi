\chapter*{Sommario}
\addcontentsline{toc}{section}{Sommario}
La diffusione e il maggior utilizzo della tecnologia \textit{blockchain} non solo come struttura per criptomonete, impone un maggior controllo sull'efficacia delle tecniche di crittografia implementate per rendere il sistema sicuro, decentralizzato e anonimo (o almeno in parte).\newline
Al fine di poter simulare, in un sistema controllato, chiuso e che permetta di realizzare alcuni dei scenari fino ad ora solo teoricamente analizzati, si utilizzerà \textit{LUNES}\cite{gdalunes}.\newline
LUNES, sviluppato Gabriele D'Angelo e Stefano Ferretti, è un simulatore altamente scalabile e modulare e permette di creare, programmare ed analizzare protocolli complessi su reti non strutturate ed estese; in particolare per le blockchain è necessario realizzare delle reti \textit{peer-to-peer} fortemente dinamiche e vaste.\newline
Lo scopo è quello di raccogliere dati e stimare delle soglie di rischio in cui per alcuni scenari l'utilizzo di una \textit{blockchain} risulti essere non sicura.\newline\newline
Seguendo un percorso suddivisibile essenzialmente in due macro fasi, la tesi si sviluppa attraverso 6 capitoli. La prima fase, dedicata all'approfondimento di tematiche di ricerca ed alla ricognizione sullo stato dell'arte, è descritta attraverso i primi 4 capitoli. Gli ultimi 2 capitoli, invece, descrivono progettazione e sviluppo degli scenari di studio e le relative analisi sui dati raccolti.\newline
In questo documento sono raccolti soltanto i capitoli che sono stati generati durante lo studio preliminare teorico e delle tecnologie di interesse del progetto di tesi. I capitoli di progettazione ed implementazione e raccolta dati (capitoli $5$ e $6$) sono lasciati per il documento finale.

\paragraph{Capitolo 1}
Il primo capitolo affronta brevemente alcuni concetti di crittografia fondamentali per capire a fondo come le blockchain garantiscono che i sistemi siano decentralizzati e sicuri; in particolare si farà riferimento alla blockchain implementata come core per i \textit{Bitcoin}.

\paragraph{Capitolo 2}
Il secondo capitolo esplora e descrive storia e caratteristiche principali della blockchain per Bitcoin per capirne le varie metodologie e scelte realizzative. Verrà presentato come \textit{Satoshi Nakamoto} ha progettato e realizzato il progetto \textit{Bitcoin} tramite consenso distribuito ed incentivi. Verranno trattati quindi temi come \textit{transazioni}, \textit{mining}, \textit{Double Spending}, \textit{fee}, \textit{Proof-of-Work} e privacy.

\paragraph{Capitolo 3}
Il terzo capitolo analizzerà i simulatori presenti e perché si è scelto di utilizzare \textit{LUNES}.
In aggiunta verrà presentato il simulatore \textit{LUNES} e come è stato utilizzato per il lavoro di tesi.

\paragraph{Capitolo 4}
Il quarto raccoglierà le varie tipologie di attacco possibili su blockchain. Da notare è che non verranno trattate vulnerabilità su applicativi relativi all'utilizzo della blockchain per le criptomonete come \textit{wallet}, \textit{exchange} o software per il \textit{mining}.

\paragraph{Capitolo 5}
Il primo capito della seconda parte del progetto di tesi presenterà come, tramite \textit{LUNES}, è stato possibile simulare alcuni dei più importanti scenari di attacco su blockchain.

\paragraph{Capitolo 6}
Il settimo capitolo raccoglierà e presenterà i dati raccolti nella precedente fase e come questi possano essere applicati al mondo reale, quali sono le soglie minime di rischio e classificare gli scenari in base al fattore di rischio.

% NeoTex: mainfile=main.tex:
