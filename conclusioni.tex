\chapter*{Conclusioni}
\addcontentsline{toc}{chapter}{Conclusioni}
L'effettiva sicurezza del protocollo Bitcoin non è sempre verificabile nella pratica. Una simulazione o un effettivo attacco sulla blockchain è molto dispendioso in termini di investimenti, risorse e tempi. Nonostante le tecnologie utilizzate nella blockchain siano già conosciute e studiate da personale esperto, non esistono figure professionali totalmente preparate ad affrontare questi tipologie di scenari. Le conoscenze necessarie per questo tipo di attività sono specialistiche e necessitano di esperienza, capacità e un nuovo modo di pensare.

Dati i risultati ottenuti per il testi di \textit{DoS} è possibile affermare la possibilità di successo di un attacco simile è quasi nulla. Un attaccante, per poter, controllare il traffico tra i \textit{peer}, deve essere in possesso di numero molto elevato di nodi; questo risulta essere sia infattibile che sconveniente. Nel caso in cui, ad esempio, un attaccante controllasse $1000$ nodi risulterebbe più conveniente agire correttamente cercando di sfruttare la potenza di calcolo per ottenere il \textit{reward} tramite mining o abbandonare l'utilizzo dei nodi come accesso alla blockchain ed utilizzarli come una \textit{botnet} per lanciare ulteriori attacchi.

Un attacco del $51\%$ non è mai stato registrato per la blockchain Bitcoin con il fine di essere sfruttato per scopi malevoli ma nel 2014 l'hashrate del \textit{pool} \textit{Ghash.IO} raggiunse questa soglia. La community, presa visione della possibile problematica, discusse su come trovare una soluzione; il \textit{pool}, volontariamente, decise di ridurre la propria potenza di calcolo. Bitcoin, data la sua vasta adozione ed una community molto numerosa potrebbe non essere mai affetta da questa problematica ma altri criptomonete come \textit{MonaCoin}, \textit{Bitcoin Gold} (una \textit{fork} di Bitcoin), \textit{zencash}, \textit{LiteCoin} ed altre minori sono state afflitte da questo problema. In questi casi si è trattato di veri e proprio attacchi in cui gli attori malevoli sono riusciti a dirottare la creazione della catena a proprio favore. Nell'attacco apportato a \textit{zencash} gli attaccanti hanno sfruttato la loro potenza di calcolo per creare delle transazioni a proprio vantaggio per il valore di $21000$ token ($\sim50000$\$). Data la popolarità delle criptomonete sono in molti a pubblicare nuove tipologie di blockchain e monete virtuali ma è proprio questa frammentazione a rendere le criptomonete minori vulnerabili a questa tipologia di attacco. Ad esempio è stato calcolato\footnote{\href{https://www.crypto51.app/}{crypto51.app}} che per ottenere un hashrate pari al $109\%$ sulla rete per \textit{MonaCoin} sono sufficienti $624$\$ (contro i $277536$\$ per l'$1\%$ su rete Bitcoin, dati aggiornati a Novembre 2018).

% NeoTex: mainfile=main.tex:
