\chapter*{Conclusioni}
\addcontentsline{toc}{chapter}{Conclusioni}
L'effettiva sicurezza del protocollo Bitcoin, così come quello applicato alle criptomonete, non è sempre esattamente verificabile nella pratica in quanto le dinamiche che regolamentano l'interazione tra i nodi, il protocollo di scambio, la verifica dei dati e la presenza di attori malevoli sono estremamente complesse e difficili da modellare in un ambiente simulato. In una simulazione spesso è doveroso trascurare alcuni dettagli al fine di rendere fattibile il progetto ma il rischio è quello di sottovalutare alcuni concetti che vanno ad influenzare la validità dei risultati. In aggiunta il campo tecnologico delle criptomonete non è ancora del tutto stato esplorato e molte strategie di progettazione, di attacco e di difesa non sono state ben definite a causa dell'alta varianza messa in gioco dalle diverse blockchain. Dal punto di vista dell'analista, invece, è opportuno anche valutare i diversi approcci che possono essere applicati tramite uno studio approfondito della teoria dei giochi applicata alla blockchain. L'applicazione di una strategia di \textit{mining} o di attacco prima non previste potrebbero distruggere una criptomoneta e minare la credibilità e la fiducia degli utenti nell'utilizzo delle altre criptomonete. È noto infatti l'alto tasso di volatilità del valore dei singoli token delle diverse monete in quanto a causa di nuovi studi, nuove blockchain, nuove implementazioni, nuovi hardware o leggi possono influenzare la fiducia degli investitori a tal punto da abbandonare un moneta per un'altra.\newline\newline

I test presentati in questo progetto di tesi non sono esaustivi per l'intero ecosistema delle blockchain ma rappresentano una macrocategoria di attacchi applicabili in molte criptomonete e che presentano molteplici sfaccettature. Il lavoro svolto in questo progetto di tesi, infatti, deve essere considerato come un punto di partenza su cui poter costruire nuove integrazioni per altre tipologie di blockchain, di protocolli e di attacchi.\newline
L'utilizzo del simulatore \textit{LUNES} ha dimostrato che è possibile modellare ed utilizzare agevolmente concetti e reti piuttosto complessi ottenendo delle performance più che accettabili.
Il linguaggio \textit{C} ha imposto una modellazione più elaborata ed una visione meno ad alto livello, a causa dell'impossibilità di applicazione di paradigmi di programmazione ad oggetti o funzionale, rendendo alcune decisioni implementative meno intuitive. La possibilità di utilizzare linguaggi di programmazione che possano mantenere un livello di dettaglio e controllo a basso livello ma al tempo stesso fornire strumenti per una migliore gestione della memoria e di costrutti ad alto livello senza sacrificare le performance (ad esempio \textit{Go} o \textit{Rust}) potrebbero incrementare l'usabilità e la possibilità di nuovi utilizzi.\newline

Nonostante le singole tecnologie utilizzate nella blockchain siano già conosciute e studiate da personale esperto, non esistono figure professionali totalmente preparate ad affrontare queste tipologie di scenari. Le conoscenze necessarie per queste di attività sono specialistiche e necessitano di esperienza, capacità e un nuovo modo di pensare. Visto il successo delle criptomonete, ma soprattutto della tecnologia \textit{blockchain}, è opportuno che il mondo dell'information technology e della computer security approfondiscano questo campo al fine di rendere più agevole, sicuro ed alla portata di tutti una tecnologia con un altissimo pontenziale.\newline
Data la popolarità delle criptomonete sono in molti a pubblicare nuove tipologie di blockchain e monete virtuali ma è proprio questa frammentazione a rendere le criptomonete minori vulnerabili a molti attacchi in cui è necessaria non solo la presenza di protocolli sicuri ma anche di utenti onesti ($51\%$).



% NeoTex: mainfile=main.tex:
